% This is LLNCS.DEM the demonstration file of
% the LaTeX macro package from Springer-Verlag
% for Lecture Notes in Computer Science,
% version 2.4 for LaTeX2e as of 16. April 2010
%
\documentclass{llncs}
%
\usepackage{amsmath}
\usepackage{pdfpages}
\usepackage{graphicx}
\usepackage{paralist}
\usepackage{xspace}
\usepackage{algorithm,algorithmic}
\newcommand{\etal}[0]{{\em et al.}\xspace}
\newcommand{\numLeaves}[0]{\ensuremath{s}\xspace}
\newcommand{\numSites}[0]{\ensuremath{n}\xspace}
\newcommand{\dataMatrix}[0]{\ensuremath{D}\xspace}
\newcommand{\tree}[0]{\ensuremath{T}\xspace}
\newcommand{\edgeLen}[1]{\ensuremath{l_{#1}}\xspace}
\newcommand{\rate}[1]{\ensuremath{r_{#1}}\xspace}
\newcommand{\ratevec}[0]{\ensuremath{\mathbf{r}}\xspace}
\newcommand{\timevec}[0]{\ensuremath{\mathbf{t}}\xspace}
\newcommand{\duration}[1]{\ensuremath{t_{#1}}\xspace}
\newcommand{\age}[1]{\ensuremath{\tau_{#1}}\xspace}
\newcommand{\parent}[1]{\ensuremath{{#1}^{\circ}}\xspace}
\newcommand{\ratePriorDensity}[0]{\ensuremath{f}\xspace}
\newcommand{\timePriorDensity}[0]{\ensuremath{g}\xspace}
\newcommand{\numAges}[0]{\ensuremath{N}\xspace}
\DeclareMathOperator*{\argmax}{\arg\!\max}
\usepackage{hyperref}
\hypersetup{backref,  linkcolor=blue, citecolor=red, colorlinks=true, hyperindex=true}
%
\begin{document}
\title{Speed dating}
\titlerunning{Speed dating}
\author{Author 1\inst{1} \and Author 2\inst{1} \and Author 3\inst{1,2}}
\authorrunning{Author 1 et al.} % abbreviated author list
\tocauthor{Author 1, Author 2}
\institute{Institute 1\\
\email{\{Authors\}@h-its.org} \and Institute 2,\\ Address, Country\\
\email{Author3@h-its.org}}
%% Eumerating parts in aligned environments %%
\newcommand\enum{\addtocounter{equation}{1}\tag{\theequation}}
\maketitle              % typeset the title of the contribution
\begin{abstract} The abstract \end{abstract}
\section {Introduction}
Akerborg \etal \cite{Akerborg2008} describe a fast method for approximating
the maximum a posteriori (MAP) estimate of the age of a node in tree.
There is not an open source implementation accompanying that paper.
Howver, these authors do distribute software primeGSR
(\url{http://prime.sbc.su.se/primeGSR/docs.html}).
That software is not explicitly mentioned in \cite{Akerborg2008}, but its manual
(\url{http://prime.sbc.su.se/primeGSR/downloads/primegsr_manual.pdf})
states ``Thanks to
new algorithms, that uses a discretization of divergence times, it can reconstruct gene trees
in time comparable to standard substitution model-based reconstruction methods.''
\section{Notation}
(trying to stick to Akerborg \etal's notation:
\begin{compactitem}
    \item[\numLeaves] is the number of leaves.
    \item[\numSites] is the number of sites
    \item[\dataMatrix] is the alignment
    \item[\tree] is the tree
    \item[``edge $v$''] means the edge connecting node $v$ to its parent.
    \item[\edgeLen{v}] is the length edge $v$ in expected number of substitutions per site
    \item[\rate{v}] is the rate of substitution along edge $v$
    \item[\duration{v}] is the duration of edge $v$ in time
\begin{equation}
    \edgeLen{v} = \rate{v}\duration{v}
\end{equation}
    \item[\age{v}] is the age of node $v$. The difference between the node time and the present (which is time 0)
    \item[\parent{v}] is the parent node of node $v$
\begin{equation}
    \duration{v} = \age{\parent{v}} - \age{v}
\end{equation}
    \item[\ratePriorDensity] is the prior probability density on rates
    \item[\timePriorDensity] is the prior probability density on durations of edges
\begin{equation}
    \mbox{MAP}(\ratevec, \timevec)  = \argmax_{\ratevec, \timevec} \Pr\left(\dataMatrix \mid \tree, \ratevec, \timevec\right) \ratePriorDensity(\ratevec) \timePriorDensity(\timevec | T)
\end{equation}
    \item[\numAges] is the discrete number of ages used in their binning
\end{compactitem}

Algorithm \ref{AMCMC} is the high level MCMC used by Akerborg \etal.
The table below describes how MTH {\em thinks} they use this to 
get the results in Figure 2.
\begin{table}
    \begin{tabular}{c|c|c|p{20em}}
\textsc{AcceptMove} & $O$ & p & Result \\
\hline
MetropHastings & posterior ratio & 0 & typical Bayesian MCMC \\
\hline
MetropHastings & posterior ratio & $>0$ & impossible (no \textsc{HastingRatioFactor} is described)\\
\hline
HillClimbing & posterior ratio & 0 & $\ratevec\times \timevec$-method for MAP \\
\hline
HillClimbing & posterior ratio & 0  & if you do not include $\ratevec$ and $\timevec$ in parameters, but just use $\edgeLen{}$ parameters, this is the $l$-method, but I don't see how they calculate the solid lines in Figure 2. Or the prior ratio for $\edgeLen$ during the MCMC. \\
\hline
HillClimbing & posterior ratio & 0.001 & combined method for MAP \\
\hline
MetropHastings & likelihood & 0 & some integrated likelihood method (not described) \\
\hline
MetropHastings & likelihood & $>0$ & impossible (no \textsc{HastingRatioFactor} is described)\\
\hline
HillClimbing & likelihood & 0 & $\ratevec\times\timevec$ optimization dashed red in Fig 2 \\
\hline
HillClimbing & posterior ratio & 0  & if you do not include $\ratevec$ and $\timevec$ in parameters, but just use $\edgeLen{}$ parameters, this is the $l$-method. dashed blue in Figure 2 \\
\hline
HillClimbing & likelihood & 0.001 & combined method targetting likelihood dotted green in figure 2 \\
\hline
\end{tabular}
\end{table}
\begin{algorithm} \caption{Combined Akerborg \etal \textsc{MCMC}}\label{AMCMC}
\begin{algorithmic}
    \REQUIRE $\boldsymbol{\theta}$, the starting values for all parameters
    \REQUIRE $p$, the probability of conducing a FactorRT proposal in any iteration
    \REQUIRE $O$, a function to calculation the target density (posterior or just the ``data likelihood''.
    \REQUIRE \textsc{AcceptMove}, a funciton that takes the proposed target density, the current target density, and the Hastings ratio for the proposal
\STATE  $\boldsymbol{\theta^{(0)}} \leftarrow \boldsymbol{\theta}$
\FOR{$i=1$ to $\infty$}
    \IF{$p == 1$ or $\mbox{Uniform}(0, 1) > p$}
        \STATE $k \sim \mbox{UniformInt}(0, \left|\boldsymbol{\theta^{(i-1)}}\right|)$
        \STATE $\theta_k^{(i-1)}\leftarrow$ element $k$ of $\boldsymbol{\theta^{(i-1)}}$
        \STATE $\theta_k^{\prime} \sim \mbox{LogNormal}(\theta_k^{(i-1)}, \sigma)$
        \STATE $q = \mbox{\textsc{HastingsRatio}}(\theta_k^{\prime}, \theta_k^{(i-1)}, \sigma)$
    \ELSE
    \STATE $\theta_k^{\prime} \rightarrow \mbox{\textsc{FactorRT}}(\boldsymbol{\theta^{(i)}})$
    \STATE $q = \mbox{\textsc{HastingsRatioFactor}}(\theta_k^{\prime}, \theta_k^{(i-1)})$
    \ENDIF
    \STATE $\boldsymbol{\theta^{\prime}}\leftarrow \boldsymbol{\theta^{(i-1)}}$ with $\theta_k^{\prime}$ substituted for parameter $k$
    \STATE $z^{\prime} = O(\theta^{\prime})$
    \IF{$\mbox{\textsc{AcceptMove}}(z^{\prime}, z^{(i-1)}, q)$}
        \STATE $z^{(i)} = z^{\prime}$
        \STATE $\boldsymbol{\theta^{(i)}} =\boldsymbol{\theta}^{\prime}$
    \ELSE
    \STATE $z^{(i)} = z^{(i-1)}$
    \STATE $\boldsymbol{\theta^{(i)}} =\boldsymbol{\theta}^{(i-1)}$
    \ENDIF
\ENDFOR
\end{algorithmic}
\end{algorithm}

\bibliographystyle{splncs03}

\bibliography{dating}
\end{document}

\begin{algorithm} \caption{}\label{}
\begin{algorithmic}
\end{algorithmic}
\end{algorithm}
