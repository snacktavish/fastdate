% This is LLNCS.DEM the demonstration file of
% the LaTeX macro package from Springer-Verlag
% for Lecture Notes in Computer Science,
% version 2.4 for LaTeX2e as of 16. April 2010
%
\documentclass{llncs}
%
\usepackage{amsmath}
\usepackage{pdfpages}
\usepackage{graphicx}
\usepackage{paralist}
\usepackage{xspace}
\usepackage{algorithm,algorithmic}
\newcommand{\etal}[0]{{\em et al.}\xspace}
\newcommand{\numLeaves}[0]{\ensuremath{s}\xspace}
\newcommand{\numSites}[0]{\ensuremath{n}\xspace}
\newcommand{\dataMatrix}[0]{\ensuremath{D}\xspace}
\newcommand{\tree}[0]{\ensuremath{T}\xspace}
\newcommand{\edgeLen}[1]{\ensuremath{l_{#1}}\xspace}
\newcommand{\rate}[1]{\ensuremath{r_{#1}}\xspace}
\newcommand{\ratevec}[0]{\ensuremath{\mathbf{r}}\xspace}
\newcommand{\timevec}[0]{\ensuremath{\mathbf{t}}\xspace}
\newcommand{\duration}[1]{\ensuremath{t_{#1}}\xspace}
\newcommand{\age}[1]{\ensuremath{\tau_{#1}}\xspace}
\newcommand{\agevec}[0]{\ensuremath{\boldsymbol{\tau}}\xspace}
\newcommand{\parent}[1]{\ensuremath{a[{#1}]}\xspace}
\newcommand{\firstChild}[1]{\ensuremath{b[{#1}]}\xspace}
\newcommand{\secondChild}[1]{\ensuremath{c[{#1}]}\xspace}
\newcommand{\subtreeOptFactor}[2]{\ensuremath{f_{[{#1}][{#2}]}}\xspace}
\newcommand{\optChildAges}[3]{\ensuremath{x_{[{#1}][{#2}][{#3}]}}\xspace}
\newcommand{\ratePriorDensity}[0]{\ensuremath{g}\xspace}
\newcommand{\timePriorDensity}[0]{\ensuremath{h}\xspace}
\newcommand{\numAges}[0]{\ensuremath{N}\xspace}
\DeclareMathOperator*{\argmax}{\arg\!\max}
\usepackage{hyperref}
\hypersetup{backref,  linkcolor=blue, citecolor=red, colorlinks=true, hyperindex=true}
%
\begin{document}
\title{Speed dating}
\titlerunning{Speed dating}
\author{Author 1\inst{1} \and Author 2\inst{1} \and Author 3\inst{1,2}}
\authorrunning{Author 1 et al.} % abbreviated author list
\tocauthor{Author 1, Author 2}
\institute{Institute 1\\
\email{\{Authors\}@h-its.org} \and Institute 2,\\ Address, Country\\
\email{Author3@h-its.org}}
%% Eumerating parts in aligned environments %%
\newcommand\enum{\addtocounter{equation}{1}\tag{\theequation}}
\maketitle              % typeset the title of the contribution
\begin{abstract} The abstract \end{abstract}
\section {Introduction}
Akerborg \etal \cite{Akerborg2008} describe a fast method for approximating
the maximum a posteriori (MAP) estimate of the age of a node in tree.
There is not an open source implementation accompanying that paper.
Howver, these authors do distribute software primeGSR
(\url{http://prime.sbc.su.se/primeGSR/docs.html}).
That software is not explicitly mentioned in \cite{Akerborg2008}, but its manual
(\url{http://prime.sbc.su.se/primeGSR/downloads/primegsr_manual.pdf})
states ``Thanks to
new algorithms, that uses a discretization of divergence times, it can reconstruct gene trees
in time comparable to standard substitution model-based reconstruction methods.''
\section{Notation}
(trying to stick to Akerborg \etal's notation:
\begin{compactitem}
    \item[\numLeaves] is the number of leaves.
    \item[\numSites] is the number of sites
    \item[\dataMatrix] is the alignment
    \item[\tree] is the tree
    \item[``edge $v$''] means the edge connecting node $v$ to its parent.
    \item[\edgeLen{v}] is the length edge $v$ in expected number of substitutions per site
    \item[\rate{v}] is the rate of substitution along edge $v$
    \item[\duration{v}] is the duration of edge $v$ in time
\begin{equation}
    \edgeLen{v} \equiv \rate{v}\duration{v}
\end{equation}
    \item[\age{v}] is the age of node $v$. The difference between the node time and the present (which is time 0)
    \item[\parent{v}] is the parent node of node $v$
\begin{equation}
    \duration{v} \equiv \age{\parent{v}} - \age{v}
\end{equation}
    \item[\firstChild{v}, \secondChild{v}] denote the two children of node $v$
    \item[\ratePriorDensity] is the prior probability density on rates
    \item[\timePriorDensity] is the prior probability density on durations of edges
\begin{equation}
    \mbox{MAP}(\ratevec, \timevec)  \equiv \argmax_{\ratevec, \timevec} \Pr\left(\dataMatrix \mid \tree, \ratevec, \timevec\right) \ratePriorDensity(\ratevec) \timePriorDensity(\timevec | T)
\end{equation}
    \item[\numAges] is the discrete number of ages used in their binning
\end{compactitem}

Algorithm \ref{AMCMC} is the high level MCMC used by Akerborg \etal.
The table below describes how MTH {\em thinks} they use this to 
get the results in Figure 2.
\begin{table}
    \begin{tabular}{c|c|c|p{20em}}
\textsc{AcceptMove} & $O$ & p & Result \\
\hline
MetropHastings & posterior ratio & 0 & typical Bayesian MCMC \\
\hline
MetropHastings & posterior ratio & $>0$ & invalid - \textsc{FactorRT} is an hill-climbing move\\
\hline
HillClimbing & posterior ratio & 0 & $\ratevec\times \timevec$-method for MAP \\
\hline
HillClimbing & posterior ratio & 0  & if you do not include $\ratevec$ and $\timevec$ in parameters, but just use $\edgeLen{}$ parameters, this is the $l$-method, but I don't see how they calculate the solid lines in Figure 2. Or the prior ratio for $\edgeLen{}$ during the MCMC. \\
\hline
HillClimbing & posterior ratio & 0.001 & combined method for MAP \\
\hline
MetropHastings & likelihood & $>0$ & invalid - \textsc{FactorRT} is an hill-climbing move\\
\hline
HillClimbing & likelihood & 0 & $\ratevec\times\timevec$ optimization dashed red in Fig 2 \\
\hline
HillClimbing & posterior ratio & 0  & if you do not include $\ratevec$ and $\timevec$ in parameters, but just use $\edgeLen{}$ parameters, this is the $l$-method. dashed blue in Figure 2 \\
\hline
HillClimbing & likelihood & 0.001 & combined method targetting likelihood dotted green in figure 2 \\
\hline
\end{tabular}
\end{table}

\begin{algorithm} \caption{Combined Akerborg \etal \textsc{MCMC}}\label{AMCMC}
\begin{algorithmic}
    \REQUIRE $\boldsymbol{\theta}$, the starting values for all parameters
    \REQUIRE $p$, the probability of conducing a FactorRT proposal in any iteration
    \REQUIRE $O$, a function to calculation the target density (posterior or just the ``data probability'').
    \REQUIRE \textsc{AcceptMove}, a funciton that takes the proposed target density, the current target density, and the Hastings ratio for the proposal
\STATE  $\boldsymbol{\theta^{(0)}} \leftarrow \boldsymbol{\theta}$
\STATE  $z^{(0)} \leftarrow O(\boldsymbol{\theta})$
\FOR{$i \in \left[1, 2,\ldots \infty \right) $}
    \IF{$\mbox{Uniform}(0, 1) > p$}
        \STATE $k \sim \mbox{UniformInt}(0, \left|\boldsymbol{\theta^{(i-1)}}\right|)$
        \STATE $\theta_k^{(i-1)}\leftarrow$ element $k$ of $\boldsymbol{\theta^{(i-1)}}$
        \STATE $\theta_k^{\prime} \sim \mbox{LogNormal}(\theta_k^{(i-1)}, \sigma)$
        \STATE $q \leftarrow  \mbox{\textsc{HastingsRatio}}(\theta_k^{\prime}, \theta_k^{(i-1)}, \sigma)$
        \STATE $\boldsymbol{\theta^{\prime}}\leftarrow \boldsymbol{\theta^{(i-1)}}$ with $\theta_k^{\prime}$ substituted for parameter $k$
        \STATE $z^{\prime} \leftarrow  O(\theta^{\prime})$
        \STATE doAccept$ \leftarrow \mbox{\textsc{AcceptMove}}(z^{\prime}, z^{(i-1)}, q)$
    \ELSE
        \STATE $\boldsymbol{\theta^{\prime}}, d \leftarrow \mbox{\textsc{FactorRT}}(\boldsymbol{\theta^{(i)}}, z^{(i-1)})$
        \STATE $z^{\prime} \leftarrow  d + z^{(i-1)}$
        \STATE doAccept$ \leftarrow$ TRUE
    \ENDIF
    \IF{doAccept}
        \STATE $z^{(i)} \leftarrow  z^{\prime}$
        \STATE $\boldsymbol{\theta^{(i)}} \leftarrow \boldsymbol{\theta}^{\prime}$
    \ELSE
    \STATE $z^{(i)} \leftarrow  z^{(i-1)}$
    \STATE $\boldsymbol{\theta^{(i)}} \leftarrow \boldsymbol{\theta}^{(i-1)}$
    \ENDIF
\ENDFOR
\end{algorithmic}
\end{algorithm}

The ``speed dating'' aspect of their work is the \textsc{FactorRT} operation.
\begin{algorithm} \caption{\textsc{FactorRT}}\label{factorRT}
\begin{algorithmic}
\REQUIRE $\ratevec$
\REQUIRE $\agevec$
\REQUIRE $\mu, \sigma^2$ the mean and variance (respectively) of the distribution of rates
\ENSURE that the returned parameter vector $\ratevec^{\prime}$ and $\agevec^{\prime}$ are
approximately optimal combinations of rates and ages that preserve $\edgeLen{}$ (and hence 
do not require recalculation of the likelihood.\\
\COMMENT{\textsc{Initialization}}
\STATE allocate \subtreeOptFactor{}{} as $2\numLeaves -1$ by $\numAges$ matrix of floating point numbers.
\STATE allocate \optChildAges{}{}{} as $2\numLeaves -1$ by $\numAges$ by $2$ matrix of integers
\FOR{$i \in \left[0, 1, \ldots, 2\numLeaves - 1\right)$}
    \STATE $\edgeLen{i} \leftarrow  \rate{i} \duration{i}$
\ENDFOR
\FOR{each leaf node, $u$, in postorder}
    \STATE \subtreeOptFactor{u}{0} = 1.0
    \FOR{$d \in [1, 2, \ldots, \numAges)$ }
        \STATE \subtreeOptFactor{u}{d} = 0.0
    \ENDFOR
\ENDFOR \\
\COMMENT{\textsc{End of Initialization}}
\FOR{each non-root, internal node $u$ in postorder}
    \FOR{$d \in [1, 2, \ldots, \numAges)$ }
        \STATE $\subtreeOptFactor{u}{d}, y, z \leftarrow \mbox{\textsc{PruneFactorRT}}(\edgeLen, \firstChild{u}, \secondChild{u}, f, d)$
        \STATE \optChildAges{u}{d}{0}, \optChildAges{u}{d}{1} = y, z
    \ENDFOR
\ENDFOR
\end{algorithmic}
\end{algorithm}



\begin{algorithm} \caption{\textsc{PruneFactorRT}}\label{pruneFactorRT}
\begin{algorithmic}
    \REQUIRE $\edgeLen{}$ the edge lengths (in expected changes per site)
\REQUIRE $v, w$ two sibling nodes
\REQUIRE $f$ The lookup table with entries already filled in for $v$ and $w$
\REQUIRE $d$ the age of the parent
\REQUIRE $\mu, \sigma^2$ the mean and variance (respectively) of the distribution of rates
\ENSURE return the highest prior density for the subtree if the parent of $v$ and $u$ is at node age $d$\\
\COMMENT{\textsc{Initialization}}
\STATE $d_v, d_w \leftarrow 0, 0$
\STATE $o = -1$ \COMMENT{impossibly low value for the best value}\\
\COMMENT{\textsc{End of Initialization}}
\FOR{$i \in [0, 1, \ldots, d)$ }
    \STATE $t_v^{\ast} \leftarrow d - i$
    \STATE $r_v^{\ast} \leftarrow l_v^{\ast} / t_v^{\ast}$
    \STATE $m_v^{\ast} \leftarrow \ratePriorDensity(r_v^{\ast})$
    \FOR{$j \in [0, 1, \ldots, d)$ }
        \STATE $t_w^{\ast} \leftarrow d - j$
        \STATE $r_w^{\ast} \leftarrow l_w^{\ast} / t_w^{\ast}$
        \STATE $m_w^{\ast} \leftarrow \ratePriorDensity(r_w^{\ast})$
        \STATE $q \leftarrow \timePriorDensity(d \mid \tree, i, j)$
        \STATE $y \leftarrow q m_v^{\ast} m_w^{\ast} \subtreeOptFactor{v}{i} \subtreeOptFactor{w}{j}$
        \IF{$y > o$}
            \STATE $o \leftarrow y$
            \STATE $d_v = i$
            \STATE $d_w = j$
        \ENDIF
    \ENDFOR
\ENDFOR
\RETURN $o, d_v, d_w$
\end{algorithmic}
\end{algorithm}

Note that the only calculation in the inner loop of \textsc{PruneFactorRT} that needs both $i$ and $j$ is the prior density of the tree.
So many of the caculations (e.g. the calculation of $m_w^{\ast} \leftarrow \ratePriorDensity(r_w^{\ast})$)
could be done in a separate loop rather than a nested loop.

For some sets of prior on divergence times, the priors on the branch durations is 
the product of each descendant branch prior so one could do two separate pruning
steps
(see \textsc{PruneFactorRTIndepPrior}) as the Pruning step.

\begin{algorithm} \caption{\textsc{PruneFactorRTIndepPrior}}\label{pruneFactorRTIndep}
\begin{algorithmic}
    \STATE $o_v, d_v\leftarrow \mbox{\textsc{PruneFactorRTIndepPriorOneChild}}(l, v, f, d)$
    \STATE $o_w, d_w\leftarrow \mbox{\textsc{PruneFactorRTIndepPriorOneChild}}(l, w, f, d)$
    \STATE $o \leftarrow o_v o_w$
    \RETURN $o, d_v, d_w$
\end{algorithmic}
\end{algorithm}

\begin{algorithm} \caption{\textsc{PruneFactorRTIndepPriorOneChild}}\label{pruneFactorRTIndepOneChild}
\begin{algorithmic}
\STATE $d_v\leftarrow 0$
\STATE $o = -1$ \COMMENT{impossibly low value for the best value}\\
\FOR{$i \in [0, 1, \ldots, d)$ }
    \STATE $t_v^{\ast} \leftarrow d - i$
    \STATE $r_v^{\ast} \leftarrow l_v^{\ast} / t_v^{\ast}$
    \STATE $m_v^{\ast} \leftarrow \ratePriorDensity(r_v^{\ast})$
    \STATE $q \leftarrow \timePriorDensity(d \mid \tree, i)$
    \STATE $y \leftarrow q m_v^{\ast}\subtreeOptFactor{v}{i}$
    \IF{$y > o$}
        \STATE $o \leftarrow y$
        \STATE $d_v = i$
        \STATE $d_w = j$
    \ENDIF
\ENDFOR
\RETURN $o, d_v$
\end{algorithmic}
\end{algorithm}
\end{document}
\COMMENT{\textsc{End of Initialization}}

\bibliographystyle{splncs03}
\bibliography{dating}
\end{document}

\begin{algorithm} \caption{}\label{}
\begin{algorithmic}
\end{algorithmic}
\end{algorithm}
