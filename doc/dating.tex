% This is LLNCS.DEM the demonstration file of
% the LaTeX macro package from Springer-Verlag
% for Lecture Notes in Computer Science,
% version 2.4 for LaTeX2e as of 16. April 2010
%
\documentclass{llncs}
%
\usepackage{amsmath}
\usepackage{pdfpages}
\usepackage{graphicx}
\usepackage{paralist}
\usepackage{xspace}
\newcommand{\etal}[0]{{\em et al.}\xspace}
\newcommand{\numLeaves}[0]{\ensuremath{s}\xspace}
\newcommand{\numSites}[0]{\ensuremath{n}\xspace}
\newcommand{\dataMatrix}[0]{\ensuremath{D}\xspace}
\newcommand{\tree}[0]{\ensuremath{T}\xspace}
\newcommand{\edgeLen}[1]{\ensuremath{l_{#1}}\xspace}
\newcommand{\rate}[1]{\ensuremath{r_{#1}}\xspace}
\newcommand{\duration}[1]{\ensuremath{t_{#1}}\xspace}
\newcommand{\age}[1]{\ensuremath{\tau_{#1}}\xspace}
\newcommand{\parent}[1]{\ensuremath{{#1}^{\circ}}\xspace}
%\newcommand{\par}[1]{\ensuremath{{#1}^{\ast}}\xspace}
\usepackage{hyperref}
\hypersetup{backref,  linkcolor=blue, citecolor=red, colorlinks=true, hyperindex=true}
%
\begin{document}
\title{Speed dating}
\titlerunning{Speed dating}
\author{Author 1\inst{1} \and Author 2\inst{1} \and Author 3\inst{1,2}}
\authorrunning{Author 1 et al.} % abbreviated author list
\tocauthor{Author 1, Author 2}
\institute{Institute 1\\
\email{\{Authors\}@h-its.org} \and Institute 2,\\ Address, Country\\
\email{Author3@h-its.org}}
%% Eumerating parts in aligned environments %%
\newcommand\enum{\addtocounter{equation}{1}\tag{\theequation}}
\maketitle              % typeset the title of the contribution
\begin{abstract} The abstract \end{abstract}
\section {Introduction}
Akerborg \etal \cite{Akerborg2008} describe a fast method for approximating
the maximum a posteriori (MAP) estimate of the age of a node in tree.
There is not an open source implementation accompanying that paper.
Howver, these authors do distribute software primeGSR
(\url{http://prime.sbc.su.se/primeGSR/docs.html}).
That software is not explicitly mentioned in \cite{Akerborg2008}, but its manual
(\url{http://prime.sbc.su.se/primeGSR/downloads/primegsr_manual.pdf})
states ``Thanks to
new algorithms, that uses a discretization of divergence times, it can reconstruct gene trees
in time comparable to standard substitution model-based reconstruction methods.''
\section{Notation}
(trying to stick to Akerborg \etal's notation:
\begin{compactitem}
    \item[\numLeaves] is the number of leaves.
    \item[\numSites] is the number of sites
    \item[\dataMatrix] is the alignment
    \item[\tree] is the tree
    \item[``edge $v$''] means the edge connecting node $v$ to its parent.
    \item[\edgeLen{v}] is the length edge $v$ in expected number of substitutions per site
    \item[\rate{v}] is the rate of substitution along edge $v$
    \item[\duration{v}] is the duration of edge $v$ in time
    \begin{equation}
        \edgeLen{v} = \rate{v}\duration{v}
    \end{equation}
    \item[\age{v}] is the age of node $v$. The difference between the node time and the present (which is time 0)
    \item[\parent{v}] is the parent node of node $v$
    \begin{equation}
        \duration{v} = \age{\parent{v}} - \age{v}
    \end{equation}
\end{compactitem}

\bibliographystyle{splncs03}
\bibliography{dating}
%
\end{document}
