% This is LLNCS.DEM the demonstration file of
% the LaTeX macro package from Springer-Verlag
% for Lecture Notes in Computer Science,
% version 2.4 for LaTeX2e as of 16. April 2010
%
\documentclass{llncs}
%
\usepackage{amsmath}
\usepackage{amssymb}
\usepackage{breqn}
\usepackage{a4wide}
\usepackage[normalem]{ulem} % for \sout to strikethrough text
\newcounter{instr}
\newcommand{\ninstr}{\refstepcounter{instr}\theinstr.}

\newcommand{\ejmcomment}[1]{{\color{green} #1}}

\newcommand{\mthcomment}[1]{{[\color{red}MTH comment: #1]}}
\newcommand{\mthstrike}[1]{{\color{red}\sout{#1}}}
\usepackage{hyperref}
\hypersetup{backref,  linkcolor=blue, citecolor=red, colorlinks=true, hyperindex=true}

\begin{document}

\title{A dynamic programming approach for speed-dating}

\titlerunning{Speed dating}

\author{Tom\'{a}\v{s} Flouri\inst{1}, Emily Jane McTavish\inst{1}, Author, Author, Author}
\authorrunning{Tom\'{a}\v{s} Flouri et al.} % abbreviated author list
\institute{Heidelberg Institute of Theoretical Studies\\
\email{Tomas.Flouri@h-its.org}}

\maketitle
%%for the moment writing up as targeted for sys bio software paper
\begin{abstract} We present a dynamic programming algorithm to rapidly
ultrametricize even very large phylogenies, implemented in the open source software package FastDate. 
This software is \ejmcomment{(will be)} capable of using node or tip dating information to scale trees to absolute time,
or estimate relative time information.
We have implemented recent developments in probabilities of trees accounting for both incomplete 
sampling of species in the present day, and the sampling in the past due to the fossilization and recovery processes.
FastDate is \ejmcomment{(will be)} available on our webserver and on github.
\end{abstract}


\section{Introduction}
As it is becoming possible to reconstruct very large phylogenies, 
researchers are interested in using these phylogenies for downstream analyses.
However, most phylogenetic analysis software provides trees with branch lengths in units of expected number of substitutions per site,
whereas downstream analyses may divergence times in years.
For many questions, such as comparisons of rates of speciation in different groups, it is necessary to have an ultrametric tree.
Phylogeographic or paleontological questions estimates in absolute time,
in order to associate evolutionary events with historical ones.

TODO Paragraph on utility of dated trees.

TODO Explanation of Akerborg GSR and other previous approaches
 \cite{Akerborg2008}

TODO Better background on branch rate distributions\\
Setting dates on a tree requires factorizing the branch lengths of an input topology
from estimates of evolutionary distances into rates and times. 
An appropriate expectation for the variation in these branch rates is difficult to select.
A common approximation is to treat branch rates as gamma distributed (REF NEEDED)

BD process background:\\
Phylogenetic trees are commonly modeled using the the constant rate birth\textendash death process (Kendall,
1948).
The tree shape is parameterized using simply a per lineage birth rate ($\lambda$) and a per lineage death rate ($\mu$).
$\lambda$ must be greater than $\mu$ or the process deterministically goes to extinction.
Recent work by Gernhard (2008) and Stadler (2009, 2010) has expanded on Kendall's basic 
formulation of the probabilities of birth\textendash death trees to incorporate several important aspects of sampling:
the effect on branch lengths of truncating the process by sampling tips at the present (Gernhard 2008),
the effect of incomplete sampling of tips at the present day (Stadler 2009), and the effect of the sampling
of fossils on recovery of ancestral taxa (Stadler 2010).
Stadler (2010, theorem 3.11) gives probability of a dated topology with $n$ leaves,
under the birth\textendash death process, with some proportion of currently sampled tips ($\rho$), a rate of recovery of taxa in the past ($\psi$),
and the standard birth and death rate parameters.


\section{Algorithm}
FastDate is based on an dynamic programming adapted from that of Akerborg \cite{Akerborg2008}.
The granularity of the dating approach can be determined by the user choosing the number of time intervals to which 
nodes can be assigned.
In short, the algorithm maximizes probability of the overall tree using \cite{Stadler2010} 's probability of sampled trees with defined node heights,
by building up a table of node height values which maximize the likelihood of the node height ($\lambda p_1(t)$), 
and the likelihood of branch lengths in the tree having a given rate based on the gamma distribution $g(r)$.

The full procedure is described in depth below.


\section {Preliminaries}
%
%
\subsection{Basic definitions}
%
A tree $T=(V,E)$ is a connected acyclic graph where $V$ is the set of {\em
nodes} and $E$ the set of {\em edges}, such that $E = V\times V$. We use the
notation $(u,v) \in E$ to denote an edge with end-points $u,v \in V$. If $T$ is
{\em oriented}, $(u,v) \in E$ indicates that the edge originates at $u$ and leads
to $v$, in which case $u$ is called the {\em parent} and $v$ the {\em child}.
Furthermore, {\em in-degree} (resp. {\em out-degree}) of a node denotes the
number of incoming (resp. outgoing edges) of $u$. In the opposite case, the
{\em degree} of node $u$ denotes the number of edges $u$ is an end-point of.
Further, we denote with $T_u$ the subtree of $T$ rooted at node $u$.  A {\em
rooted binary tree} $T$ is a {\em directed} tree with all nodes having
in-degree 1 and out-degree 1 ({\em inner} nodes), or in-degree 1 and out-degree
0 ({\em leaves}). Furthermore, one node has in-degree 0 and out-degree 2 ({\em
root}).  In the rest of the text we implicitly assume under the term tree a
binary rooted tree.  We denote the set of leaves of tree $T$ as $L(T)$.  The
cardinality of a set $X$ is denoted as $|X|$. The {\em height} $h(u)$ of a node
$u$ of $T$ is defined as 
%
\[ h(u) = \left\{ \begin{array}{ll}
\max(h(v), h(w)) + 1 & \quad : \quad u \notin L(T)\\
1                    & \quad : \quad u    \in L(T)\\
\end{array}\right. \] 
The height $h(T)$ of a tree $T$ is the height of its root.: Finally, we will
use implicitly the notation $\ell_{u,v}$ for the length of edge $(u,v)$.

\subsection{Branch rate distribution}
\subsection{Gamma function, Gamma distribution}

The {\em Gamma function} is an extension to the factorial function to handle
any real or complex number as argument. For point $t$ it is defined as
$$\Gamma(t) = \int_0^\infty x^{t-1} e^{-x} dx.$$

The {\em probability density function} in the shape-rate parameterization of a
{\em gamma distribution} given the {\em mean} $\bar{r}$ and variance $\sigma^2$ 
\sout{\mthcomment{By convention, $\sigma$ should mean the square root of the variance}}
is
$$ \gamma(x;\alpha,\beta) = \frac{\beta^{\alpha}x^{a-1}e^{-x\beta}}{\Gamma(\alpha)} $$

where $\beta = \bar{r} / \sigma^2$ and $\alpha = \bar{r} \beta$.
\ejmcomment{I re-labeled this with $\bar{r}$ for the mean of the gamma prior to avoid confusion with $\mu$ for birth rate.}

Its natural logarithm can be computed as
$$ \ln \gamma(x;\alpha,\beta) = \alpha\ln\beta + (\alpha-1)\ln x -x\beta - \ln\Gamma(\alpha) $$

\subsection{Birth-death process}

We assume two extended types of the {\em constant rate birth-death process}
\cite{Kendall1948} that take into account individuals sampled through time. The
first one takes into account sampling of only extant individuals
\cite{Stadler2009} while the second, which is a generalization of the first,
takes into account extinct individuals \cite{Stadler2010}.

Given a constant birth rate $\lambda$ and a constant death rate $\mu$ such that
$0 \leq \mu < \lambda$. An individual at the present is sampled with
probability $\rho$.  The probability density of a sampled tree $T$ with $n$
extant sampled leaves is given by the formula
%
%
$$f(T|n) = n(\lambda-\mu)\frac{e^{-(\lambda-\mu)x_1}}{\rho\lambda + (\lambda(1 -\rho)-\mu)e^{-(\lambda-\mu)x_1}}\prod_{i=1}^{n-1}
\frac{\lambda\rho(\lambda-\mu)^2e^{-(\lambda-\mu)x_i}}{(\rho\lambda + (\lambda(1-\rho)-\mu)e^{-(\lambda-\mu)x_i})^2}.$$
%
%
The probability density of a sampled tree $T$ with $n$ extant sampled leaves,
$m$ extinct sampled leaves, $n+m > 0$, and $k \geq 0$ sampled individuals with
sampled descendants, conditioned on sampling $n$ present day individuals is
%
%
$$f[T|n,m,k] = \frac{4n\rho\lambda\psi^{k+m}}{c_1(c_2+1)(1-c_2+(1+c_2)e^{c_1x_1})}\prod_{i=1}^{n+m-1}\lambda p_1(x_i)\prod_{i=1}^{m}\frac{p_0(y_i)}{p_1(y_i)}$$
%
%N.B. There is a typo in this equation in \cite{Stadler2010} but logic and Stadler say that last term in denominator is $e^{c_1x_1}$ not  $e^{c_1x}$ \\
%
where $p_0(t)$ is the probability that an individual alive at time $t$ before
present has no sampled extinct or extant descendants, and $p_1(t)$ is the
probability that an individual alive at time $t$ before present has precisely
one sampled extant descendant and no sampled extinct descendant. Those probabilities
are defined as
%
$$p_0(t) = \frac{\lambda+\mu+\psi+c_1\frac{e^{-c_1 t}(1-c_2)-(1+c_2)}{e^{-c_1t}(1-c_2)+(1+c_2)}}{2\lambda}$$
and
$$p_1(t) = \frac{4\rho}{2(1-c_2^2)+e^{-c_1t}(1-c_2)^2+e^{c_1t}(1+c_2)^2}$$
%
and the constants $c_1$ and $c_2$ are defined as
%
$$c_1 = \sqrt{(\lambda-\mu-\psi)^2 + 4\lambda\psi} \qquad c_2 = \frac{\lambda-\mu-2\lambda\rho-\psi}{c_1}$$
%
For more information see \cite{Stadler2010} (Section 3).

%%%%% TODO: Move the following to the DP section %%%%%
%%%%%\begin{enumerate}
%%%%%\item[$\lambda$]  is the per lineage birth (speciation) rate
%%%%%\item[$\mu$]  is the per lineage death (extinction) rate
%%%%%\item[$\psi$]  is the per lineage rate of sampling in the past (e.g. fossils) (in same units as $\lambda$ and $\mu$).
%%%%%\item[$\rho$ ] is the proportion of total extant descendants $N$ which are sampled $n$.
%%%%%\item[$n$] is the number of extant (current) sampled descendants \mthstrike{of a node}.
%%%%%\item[$m$] is the number of extinct (fossil) sampled descendants \mthstrike{of a node}.
%%%%%\item[$c_1,c_2$]  are useful constants.
%%%%%$$c_1 = |\sqrt{(\lambda-\mu-\psi)^2 + 4\lambda\psi}|$$
%%%%%$$c_2 = \frac{\lambda-\mu-2\lambda\rho-\psi}{c_1}$$
%%%%%\end{enumerate}
%%%%%
%%%%%$p_1(t)$ is the probability that an individual alive at time $t$ before today has precisely 1 sampled extant descendants and no sampled extinct descendants.
%%%%%$${p_1}(t) = \frac{4\rho}{2(1-c_2^2)+e^{-c_1t}(1-c_2)^2+e^{c_1t}(1+c_2)^2}$$
%%%%%%And with no sampled fossils ($\psi=0$)
%%%%%%$$f[T|t_{mrca}=x_1,n] = n(\lambda - \mu) \frac{e^{-(\lambda-\mu)x_1}}{\rho\lambda + (\lambda(1-\rho)-\mu)e^{-(\lambda-\mu)x_1}}\prod_{i=1}^{n-1}\frac{\lambda\rho(\lambda-\mu)^2e^{-(\lambda-\mu)x_1}}{(\rho\lambda + (\lambda(1-\rho)-\mu)e^{-(\lambda-\mu)x_1})^2}$$\\
%%%%%
%%%%%For convenience in the algorithm we refer to the first part of $F[T|n]$ as $s(t,n)$.
%%%%%$$s(t,n) = \frac{4n\rho\lambda\psi^{k+m}}{c_1(c_2+1)(1-c_2+(1+c_2)e^{c_1x_1})}$$
%%%%%
%%%%%and the second part as $p_c(v)$.
%%%%%$$p_{c}(v) = \prod_{i=1}^{L(v)-1}\lambda p_1(x_i)$$
%%%%%This is convenient, as we can build up the partial products of the second part of the birth death
%%%%%probability density can be built up as nodes are traversed by
%%%%%the dynamic programming algorithm, and stored for the calculation of $F[T|n]$ at each node.
%%%%%
%%%%%\ejmcomment{Still need to ensure that this is actually how this simplifies out when psi=0}

\section{Dynamic programming algorithm}

We present the dynamic programming algorithm for relative dating of a tree
$T=(V,E)$. This will be extended in the next section to accommodate fossil
information.

We are given an integer $N \geq h(T)$ which we use to construct a grid of
time-lines numbered from 1 to $N$, such that the (uniform) distance between any
two adjacent lines is $1 / (N-1)$ and line $i$ represents time
$\frac{i-1}{N-1}$ in relative time units. The goal is to project each node of
$T$ onto one of the $N$ time-lines using a mapping $\phi : V \mapsto
\{1,\ldots,N\}$ such that a) the score of the formula
%
%
\begin{equation}\label{eq:score}
\begin{split}
f(T\ |\ n,\phi) = & n (\lambda-\mu)
                    \frac {(\lambda\rho)^{n-1} (\lambda-\mu)^{2n-2}
                           e^{-(\lambda-\mu)\tau_r}}
                          {\rho\lambda + 
                           (\lambda(1 -\rho)-\mu)
                           e^{-(\lambda-\mu)\tau_r}} \times  \\
                  & \prod_{u\notin L(T)}\frac{e^{-(\lambda-\mu)\tau_u}}
                                             {(\rho\lambda +
                                              (\lambda(1-\rho)-\mu)
                                              e^{-(\lambda-\mu)\tau_u})^2}
                                        \gamma(\frac{\ell_{u,v}}{\tau_u-\tau_v})
                                        \gamma(\frac{\ell_{u,w}}{\tau_u-\tau_w}).
\end{split}
\end{equation}
%
%
is maximized and b) the following two properties are maintained
%
%
\begin{enumerate}
\item $\forall u,v,w \in V : (u,v), (u,w) \in E \Rightarrow \phi(u) > \phi(v)%
                                                     \wedge \phi(u) > \phi(w)$,
\item $\phi(u) = 1, \forall u \in L(T)$.
\end{enumerate}
%
%
The term $f(T\ |\ n,\phi)$ is the score of tree $T$ for the mapping $\phi$
given the parameters of the birth-death process gamma distribution for
obtaining edge rate probability densities.  The age $\tau_u$ of each inner node
$u$ is computed as $\tau_v = \frac{\phi(u)-1}{N-1}$ and its two edge rates as
$\ell_{u,v}/(\tau_u - \tau_v)$ and $\ell_{u,w}/(\tau_u-\tau_w)$, where $v$ and
$w$ are the two children of $u$.
%
The restriction imposed by Property 1 ensures that no node $v$ in subtree $T_u$
is placed on time-line $\phi(v) \geq \phi(u)$. Property 2 maps all leaves onto
the first time-line.
%
%
%%%\ejmcomment{Not true in case of tip dating, but still currently true. 
%%%Also - we have some inconsistency on whether they are mapped to line 0 or 1.
%%%In the code it is 1 currently.}
%%% TOMAS: For simplicity, I changed the section to describe relative dating only and then
%%% extend it in the next section with fossil information. Therefore, I left property 2 intact.

Before describing the algorithm, it is necesary to define the interval of
discretized time-lines a node can be placed on, such that Property 1 holds. The
interval $d(r)$ for root $r$ is defined as $$d(r) = \{ i\ |\ h(r) \leq i <
N\}$$ and for an arbitrary node $u$ as $$d(u) = \{ i\ |\ h(u) \leq i < \max
d(p); (p,u) \in E\}.$$ This is justified by the fact that the farthest {\em
path} from node $u$ leading to a leaf is of size $h(u)$ and contains exactly
$h(u)$ nodes (including $u$).  Therefore, we require at least $h(u)$ time-lines
for placing each node of the path to a distinct time-line.  Hence, the first
time-line $u$ can be placed on is $h(u)$. The same applies for the path from
$u$ leading to the root node $r$.  Each node must be placed on a distinct
time-line, and as such, the last time-line $u$ may be placed on is one before
the last time-line its parent node $p$ can be placed on. The intervals for each
node can be computed by a preorder traversal of the tree. One method to achieve
maximization of (\ref{eq:score}) is to split $f(T\ |\ n, \phi)$ into two terms
$f_1(T\ |\ n,\phi(r))$ and $f_2(T\ |\ \phi)$
%
%
\begin{equation}\label{eq:part1} 
f_1(T\ |\ n,\phi(r)) = n (\lambda-\mu)
                       \frac{(\lambda\rho)^{n-1}% 
                             (\lambda-\mu)^{2n-2}% 
                             e^{-(\lambda-\mu)\tau_r}}%
                            {\rho\lambda +%
                             (\lambda(1 -\rho)-\mu)%
                             e^{-(\lambda-\mu)\tau_r}}
\end{equation}
%
%
\begin{equation}\label{eq:part2}
f_2(T\ |\ \phi) = \prod_{u\notin L(T)}
                  \frac{e^{-(\lambda-\mu)\tau_u}}
                       {(\rho\lambda + 
                        (\lambda(1-\rho)-\mu)
                        e^{-(\lambda-\mu)\tau_u})^2}
                  \gamma(\frac{\ell_{u,v}}{\tau_u-\tau_v})
                  \gamma(\frac{\ell_{u,w}}{\tau_u-\tau_w}).
\end{equation}
%
%
and compute the $|d(r)|$ maximized values of $f_2(T\ |\ \phi_i), h(r) \leq i
\leq N$, i.e. for each time-line the root may be placed on. Because $f_1(T\ |\
n,\phi)$ depends only on the root time-line, the optimal mapping $\phi_i$ that
maximizes \ref{eq:score} (denoted by $\hat\phi$) is the one that leads to the
highest product $f1_(T\ |\ n,\phi_i)\times f_2(T\ |\ \phi_i)$.%
Our approach is to maximize term \ref{eq:part2} using a dynamic programming
algorithm. For that, we slightly modify the structure of \ref{eq:part2} into
the following equivalent formula, 
%
%
\begin{equation}\label{eq:DP}
\begin{split}
s(u\notin L(T),i) = & \overbrace{\frac{e^{-(\lambda-\mu)\tau_u}}
                                      {(\rho\lambda + 
                                       (\lambda(1-\rho)-\mu)
                                       e^{-(\lambda-\mu)\tau_u})^2}
                      }^{\textrm{birth-death term } b(i)}\times\\
                    & \underbrace{
                        \max\{ s(v,j)\gamma(\frac{(N-1)\ell_{u,v}}{i-j})\ |\ 
                          h(v) \leq j < i\}
                      }_{\textrm{max score of left child}}\times \\
                    & \underbrace{
                        \max\{ s(w,k)\gamma(\frac{(N-1)\ell_{u,w}}{i-k})\ |\ 
                          h(w) \leq k < i\}
                      }_{\textrm{max score of right child}}\\
\end{split}
\end{equation}
%
%
where $s(u\in L(T),1) = 1$.
%
%
The DP traverses the target tree in postorder and for each inner node $u$, it
computes $|d(u)|$ maximized values $s(u,i), h(u) \leq i < h(u) + |d(u)|$, one
for each time-line that node $u$ may be placed on.  The score $s(u,i)$ of a
particular time-line $i$ of node $u$ is computed by finding the time-lines $j$
resp. $k$ of its two children $v$ resp. $w$, which maximize the products of
$s(u,j)$ resp. $s(w,k)$ times the probability densities of the factorized edge
priors $\gamma(\frac{(N-1)l_{u,v}}{i-j})$ resp.
$\gamma(\frac{(N-1)l_{u,w}}{i-k})$.  Note that, for a pair of time-lines $i$
and $j$ resp. $k$, we can factorize the branch length of edge $(u,v)$ resp.
$(u,w)$ into a product of time (age) and rate.
%
%
This results into ${\cal O}((|d(v)|+|d(w)|)\times|d(u)|)$ computations for
determining the $|d(u)|$ maximal scores (and mappings) for $T_u$. These maximal
values for each time-line are then stored in a table together with the
positions (time-lines) its two direct descendants (children) were placed on.
This procedure is repeated recurrently for each node in the tree starting from
the leaves and moving towards the root (postorder).Finally, once the terms
$s(r,i)$ are computed for each time-line of the root node, they are multiplied
with the corresponding term $f_1(T\ |\ n, \phi(r))$, and the time-line yielding
the maximal score is selected.

At this point, the process of backtracking starts, with the goal of retrieving
the optimal mapping $\hat\phi$. By selecting the maximal score (and hence the
time-line of the root) we also obtain the time-lines of its two children, and,
at each of the two children, the time-lines of their two children. Recursively,
we can obtain the time-lines of all nodes that maximize term \ref{eq:score}.
%
Fig.~\ref{fig:dp} describes the algorithm in detail.

\setcounter{instr}{0}
\begin{figure}[t]
\begin{center}
\begin{tabular}{|rl|}
\hline
\multicolumn{2}{|l|}{\textsc{DP}$(T, N, u)$}\\
\ninstr & $v \leftarrow$ left child of $u$\\
\ninstr & $w \leftarrow$ right child of $u$\\
\ninstr & $r \leftarrow$ the root of $T$\\
\ninstr & \textbf{if} $v \in L(T)$ \textbf{then}\\
\ninstr & \qquad $M(v,1) \leftarrow 1$\\
\ninstr & \qquad \textbf{return}\\
\ninstr & \textsc{DP}(T,N,v)\\        
\ninstr & \textbf{if} $w \in L(T)$ \textbf{then}\\
\ninstr & \qquad $M(w,1) \leftarrow 1$\\
\ninstr & \qquad \textbf{return}\\
\ninstr & \textsc{DP}(T,N,w)\\        
\ninstr & \textbf{for} $i \leftarrow \min d(u)$ \textbf{to} $\max d(u)$ \\
\ninstr & \qquad $\hat{f}_u \leftarrow 0, \hat{t}_u \leftarrow (i-1)/(N-1)$\\
\ninstr & \qquad \textbf{for} $j \leftarrow \min d(v)$ \textbf{to} $i-1$ \\
\ninstr & \qquad \qquad  $\tau_v \leftarrow (j-1)/(N-1), r_{u,v} \leftarrow \ell_{u,v}/(\tau_u - \tau_v)$ \\
\ninstr & \qquad \qquad  \textbf{if} $\gamma(r_{u,v})  s(v,j) > \hat{f}_v$ \textbf{then} \\
\ninstr & \qquad \qquad \qquad $\hat{f}_v \leftarrow \gamma(r_{u,v})  s(v,j)$\\
\ninstr & \qquad \qquad \qquad $\hat j \leftarrow j$\\
\ninstr & \qquad \textbf{for} $k \leftarrow \min d(w)$ \textbf{to} $i-1$ \\
\ninstr & \qquad \qquad  $\tau_w \leftarrow (k-1)/(N-1), r_{u,w} \leftarrow \ell_{u,w}/(\tau_u - \tau_w)$\\
\ninstr & \qquad \qquad  \textbf{if} $\gamma(r_{u,w})  s(v,k) > \hat{f}_w$ \textbf{then} \\
\ninstr & \qquad \qquad \qquad $\hat{f}_w \leftarrow \gamma(r_{u,w})  s(w,k)$\\
\ninstr & \qquad \qquad \qquad $\hat k \leftarrow k$\\
\ninstr & \qquad $M(u,i) \leftarrow b(i) \times \hat f_{v} \times \hat f_{w}$\\
\ninstr & \qquad $M^l(u,i) \leftarrow \hat j, M^r(u,i) \leftarrow \hat k$\\
\ninstr & \qquad \textbf{if} $u = r$  \textbf{then}\\
\ninstr & \qquad \qquad $M(u,i) \leftarrow M(u,i) \times f_1(T\ |\ n,i)$ \\

\hline  
\end{tabular}
\end{center}
\caption{The DP algorithm for computing relative divergence times. The
algorithm starts by passing the tree ($T$), number of discretization lines
($N$) and the root of $T$ ($u$). It recursively traverses the tree in
postorder and then computes the best placement for every node $u$.}
\label{fig:dp}
\end{figure}

W.l.o.g we prove that the DP algorithm computes the optimal score of
ref{formula} given the grid size and the values of $\lambda,\mu,\rho,\psi$.

\begin{theorem}[Optimality of DP]
Given the number of discretization intervals, the values of $\lambda,\mu,\rho$,
Algorithm DP computes the optimal score of (\ref{eq:score}).
\end{theorem}
\begin{proof}
We show that the score terms $s(r,i), h(r) \leq i \leq N$ the DP computes are
in fact equal to $f_2(T,\hat\phi_i)$ and hence, finding the product
$f_2(T,\phi_i)\times f_1(T,\phi_i)$ yields the highest value gives us the
optiomal solution. We show our proof by strong induction on the height of the
root node.
\paragraph{Base case.} Let $u$ be the parent of $v$ and $w$, such that $h(u) =
1$. Nodes $v$ and $w$ are therefore leaves, which leads us to compute
$|d(u)|=N-1$ values, of the form
%
%
$$
s(u,i) = \frac{e^{-(\lambda - \mu)\tau_u}}
              {(\rho\lambda + (\lambda(1 - \rho) - \mu)
                e^{-(\lambda-\mu)\tau_u})^2}
         \gamma(\frac{\ell_{u,v}}{\tau_u})
         \gamma(\frac{\ell_{u,w}}{\tau_u})\\
$$
%
%
for each $i \in d(u)$. Since DP selects the score that maximizes the product
when multiplied with $f_1(T,\phi)$, we obtain the optimal score.
\paragraph{Inductive step.} Assume the claim holds for each node $u$ of height
$1 \leq h(u) \leq m$. Our task is to prove that the claim holds for $h(u) =
m+1$.  According to the definition of height, a node $u$ of height $h(u)=m+1$
has a child $v$ of height $h(v)=m$ and a child $w$ of height $0 \leq h(w) \leq
m$.  Since the values $s(v,j), h(v) \leq j \leq \max d(v)$ and $s(w,k), 1 \leq
k \leq \max d(w)$ are maximized (assumption), we compute $|d(u)|$ values
%
%
\begin{equation*}
\begin{split}
s(u,i) = & \frac{e^{-(\lambda-\mu)\tau_u}}
                           {(\rho\lambda + 
                            (\lambda(1-\rho)-\mu)
                            e^{-(\lambda-\mu)\tau_u})^2}\times \\
         & \max\{ s(v,j)\gamma(\frac{(N-1)\ell_{u,v}}{i-j})\ |\ 
               h(v) \leq j < i\} \times \\
         & \max\{ s(w,k)\gamma(\frac{(N-1)\ell_{u,w}}{i-k})\ |\ 
               h(w) \leq k < i\}
\end{split}
\end{equation*}
%
%
and pick the maximal.
\qed\end{proof}

\section{Extending the DP algorithm with fossil information}


{\bf TOMAS: I moved the following text in a new section as a base for integrating fossils}

The score function we compute is placed 
%
%
%

Next, we define the maximization function $f$ that is computed for each node $u \in
V\setminus L(T)$ and for each discretization line $t_u$:  
$$f(u,t_u) = \lambda p_1(t_u) \hat{f}_v \hat{f}_w $$\\ 
where $t_u \in d(u)$, $t_v \in d(v)$, $t_w \in
d(w)$.
$\hat{f}_v$ is the maximum product of $g(r_{u,w})$ and $p_c(w,t_w)$
over the possible positions of node v given the placement of u.

Therefore the full value being maximized at each discretization line for the node $u$
is 
$$f(u,t_u) = \lambda p_1(t_u) p_c(v,t_v) g(r_{u,v}) p_c(w,t_w) g(r_{u,w})$$
rearranged
$$f(u,t_u) = \lambda p_1(t_u) p_c(v,t_v)  p_c(w,t_w) g(r_{u,v}) g(r_{u,w})$$
simplified
$$f(u,t_u) = p_c(t_u) g(r_{u,v}) g(r_{u,w})$$

To calculate the position of the root (node $r$) it is necessary to maximize the
full birth death process probabilities the branch rate probabilities.
$$f(r,t_r) \leftarrow s(r,n) p_c(t_{r}) \hat g(r_{r,v}) \hat g(r_{r,w})$$

\ejmcomment{and that last term is already maximized - proof to come}

To incorporate priors on node times (e.g. from fossil calibrations)
requires only adding a term to $f(u,t_u)$ that weights the prior probability of 
node $u$ being placed on line $t_u$, $Pr(u,t_u)$

With node priors:
$$f(u,t_u) = p_c(t_u) g(r_{u,v}) g(r_{u,w}) Pr(u,t_u)$$

Tip dating changes the assignments of node heights, and therefore 
the set of potential discretization lines on which nodes can be placed,
but does not otherwise affect the algorithm.

Once the optimal position of the root has been determined,
then backtracking through maximal values stored for the 
left and right children will give a fully specified dated tree.


\section{Description of software}
FastDate requires as input a fully bifurcating rooted phylogeny with branch lengths,
and user defined estimates of birth rate ($\lambda$), death rate ($\mu$), 
and proportion of descendants of the most recent common ancestor of the taxa found in the tree
which were sampled in the tree ($\rho$).
Optional parameters include $\psi$ the rate of recovery of fossils per lineage.



\section{Example}
\subsection{Node dating example from 1 KITE}

\subsection{Tip dating example from Stadler and Yang}

\section{Discussion}


\bibliographystyle{splncs03}
\bibliography{dating}
\end{document}
